%%
%% Beginning of file 'main.tex'
%%
%% using aastex version 6.2
\documentclass{aastex62}

\newcommand{\vdag}{(v)^\dagger}
\newcommand\aastex{AAS\TeX}
\newcommand\latex{La\TeX}

\usepackage{float}
\usepackage{longtable}

%% Tells LaTeX to search for image files in the 
%% current directory as well as in the figures/ folder.
\graphicspath{{./}{figures/}}

%% Reintroduced the \received and \accepted commands from AASTeX v5.2
%\received{January 1, 2018}
%\revised{January 7, 2018}
%\accepted{\today}
%% Command to document which AAS Journal the manuscript was submitted to.
%% Adds "Submitted to " the arguement.
%\submitjournal{ApJ}

%% Mark up commands to limit the number of authors on the front page.
%% Note that in AASTeX v6.2 a \collaboration call (see below) counts as
%% an author in this case.
%
%\AuthorCollaborationLimit=3
%
%% Will only show Schwarz, Muench and "the AAS Journals Data Scientist 
%% collaboration" on the front page of this example manuscript.
%%
%% Note that all of the author will be shown in the published article.
%% This feature is meant to be used prior to acceptance to make the
%% front end of a long author article more manageable. Please do not use
%% this functionality for manuscripts with less than 20 authors. Conversely,
%% please do use this when the number of authors exceeds 40.
%%
%% Use \allauthors at the manuscript end to show the full author list.
%% This command should only be used with \AuthorCollaborationLimit is used.

%% The following command can be used to set the latex table counters.  It
%% is needed in this document because it uses a mix of latex tabular and
%% AASTeX deluxetables.  In general it should not be needed.
%\setcounter{table}{1}

%%%%%%%%%%%%%%%%%%%%%%%%%%%%%%%%%%%%%%%%%%%%%%%%%%%%%%%%%%%%%%%%%%%%%%%%%%%%%%%%
%%
%% The following section outlines numerous optional output that
%% can be displayed in the front matter or as running meta-data.
%%
%% If you wish, you may supply running head information, although
%% this information may be modified by the editorial offices.
\shorttitle{Searching for Lyman Continuum}
\shortauthors{Dustin Davis}
%%
%% You can add a light gray and diagonal water-mark to the first page 
%% with this command:
% \watermark{text}
%% where "text", e.g. DRAFT, is the text to appear.  If the text is 
%% long you can control the water-mark size with:
%  \setwatermarkfontsize{dimension}
%% where dimension is any recognized LaTeX dimension, e.g. pt, in, etc.
%%
%%%%%%%%%%%%%%%%%%%%%%%%%%%%%%%%%%%%%%%%%%%%%%%%%%%%%%%%%%%%%%%%%%%%%%%%%%%%%%%%

%% This is the end of the preamble.  Indicate the beginning of the
%% manuscript itself with \begin{document}.

\begin{document}

\title{Dustin Davis Second Year Project: Searching for Lyman Continuum with HETDEX Data Release 1 - Preliminary Results}

%% LaTeX will automatically break titles if they run longer than
%% one line. However, you may use \\ to force a line break if
%% you desire. In v6.2 you can include a footnote in the title.

%% A significant change from earlier AASTEX versions is in the structure for 
%% calling author and affilations. The change was necessary to implement 
%% autoindexing of affilations which prior was a manual process that could 
%% easily be tedious in large author manuscripts.
%%
%% The \author command is the same as before except it now takes an optional
%% arguement which is the 16 digit ORCID. The syntax is:
%% \author[xxxx-xxxx-xxxx-xxxx]{Author Name}
%%
%% This will hyperlink the author name to the author's ORCID page. Note that
%% during compilation, LaTeX will do some limited checking of the format of
%% the ID to make sure it is valid.
%%
%% Use \affiliation for affiliation information. The old \affil is now aliased
%% to \affiliation. AASTeX v6.2 will automatically index these in the header.
%% When a duplicate is found its index will be the same as its previous entry.
%%
%% Note that \altaffilmark and \altaffiltext have been removed and thus 
%% can not be used to document secondary affiliations. If they are used latex
%% will issue a specific error message and quit. Please use multiple 
%% \affiliation calls for to document more than one affiliation.
%%
%% The new \altaffiliation can be used to indicate some secondary information
%% such as fellowships. This command produces a non-numeric footnote that is
%% set away from the numeric \affiliation footnotes.  NOTE that if an
%% \altaffiliation command is used it must come BEFORE the \affiliation call,
%% right after the \author command, in order to place the footnotes in
%% the proper location.
%%
%% Use \email to set provide email addresses. Each \email will appear on its
%% own line so you can put multiple email address in one \email call. A new
%% \correspondingauthor command is available in V6.2 to identify the
%% corresponding author of the manuscript. It is the author's responsibility
%% to make sure this name is also in the author list.
%%
%% While authors can be grouped inside the same \author and \affiliation
%% commands it is better to have a single author for each. This allows for
%% one to exploit all the new benefits and should make book-keeping easier.
%%
%% If done correctly the peer review system will be able to
%% automatically put the author and affiliation information from the manuscript
%% and save the corresponding author the trouble of entering it by hand.


\email{dustin.davis@utexas.edu}

\author[0000-0002-8925-9769]{Dustin Davis}
\affil{University of Texas at Austin \\
TBD Addr \\
Austin, TX}


\begin{abstract}

{ \color{red} todo: Investigate spectral features, specifically looking for LyC, in stacked clean sample of HDR1 LyA detections. Compare to Steidel 2018.}

\end{abstract}

%% Keywords should appear after the \end{abstract} command. 
%% See the online documentation for the full list of available subject
%% keywords and the rules for their use.
\keywords{Lyman Continuum, HETDEX}



\section{Introduction} \label{sec:intro}

By approximately 400,000 years into its life, the universe had expanded (and cooled) sufficiently to allow the naked protons created in the Big Bang to capture electrons and form neutral atoms (mostly hydrogen). As the universe continued to expand and cool, clumps of gas began to accumulate in dark matter halos, eventually giving birth to the first (Population I) stars (z $\sim$ 20-30) and later still the first galaxies (z $\sim$ 10) \cite{Bromm}. High energy photons, generally from massive main sequence stars, supernovae, and quasars, began filling the universe and, as they encountered neutral atoms, stripped away their electrons, (re)ionizing them. By z $\sim$ 7, the IGM is almost fully ionized (though there remain scattered clouds of neutral gas) \cite{Stark}.\\

While it is clear that reionization requires copious numbers of ionizing photons, and those photons are produced by O and B stars, supernovae, and quasars it is unclear to what degree each of these mechanisms contribute such photons to the IGM. Quasars are fantastically bright (across the electromagnetic spectrum) and they are certainly active during reionization, but they are likely too few in number to dominate the ionizing photon budget \cite{Bromm}. Supernovae too are extremely bright, but while more numerous than quasars, are extremely transient and, though they may turn out to aid in the escape of UV photons from the galaxies, they cannot be a principle contributor to reionization. So, unless we turn to more speculative physics (self-interacting dark matter, cosmic strings, mini-quasars, etc), we are left with O/B stars as the primary driver of reionization, but there is a problem.\\

Certainly, star forming galaxies internally produce more than enough ionizing photons to fully account for IGM reionization, but, to succinctly state the problem, the fraction of UV photons that (on average) escape the host galaxies ($f^{UV}_{esc}$ or just $f_{esc}$ hereafter) into the IGM is extremely model dependent and effectively unconstrained. The predicted escape fraction depends (sometimes very sensitively) on a myriad of parameters (some poorly constrained) including the host galaxy's mass, dust content (and related mean metallicity), and neutral hydrogen column density and covering fraction. These individual predictions vary wildly from well less than 1\% to nearly 100\% \cite{Zackrisson} though the mean is probably on the lower end. For galaxies, essentially alone, to reionize the universe by z $\sim$ 7, most of the photons would have to come from smaller, fainter galaxies and the faint-end slope of the luminosity function would need to continue down to $M_{UV} \approx -13$ with $f_{esc} \approx 0.13 - 0.20$. If the faint-end slope flattens before then, $f_{esc}$ would need to be greater. Current observations reveal no turn-over down to $M_{UV} \approx -17$ with limited statistics suggesting the slope \textit{might} continue to steepen down to $M_{UV} \approx -14$ \cite{Livermore}. What ever the case, we must combine this with knowledge of the UV escape fraction if we are to have any real hope of understanding the mechanisms of reionization.\\

Ideally, we would resolve this issue by directly measuring $f_{esc}$, but as we push to higher redshifts, photons with wavelengths shorter than Ly$\alpha$ (1216$\AA$) and especially the hard ionizing photons we wish to measure (short-ward of the Lyman Limit (912 $\AA$)) are seriously attenuated by clouds of neutral hydrogen along our line of sight such that, in the most optimistic case, by z=4 (well short of our target of $7 < z < 9$) the signal is gone. Though it would be impossible to directly measure $f_{esc}$, all hope is not lost if it is possible to accurately estimate $f_{esc}$ from other measurements that can actually be made.\\

{ \color{red} todo: competing theory as to what reionized universe ... big galaxies (but takes time, to build, so start slow and ramp up quickly) or small galaxies ... many more but with few stars but better escape fraction). If we can constrain the escape fraction we could support or eliminate one or other option ... cite Finkelstein 2019}\\

{ \color{red} todo: Make a brief case for z 3-3.5 as better proxies for z 7-10 galaxies than z=0 dwarfs, etc}\\

{ \color{red} todo: Brief description of HETDEX, etc (maybe needs to be in another section?}\\

{ \color{red} todo:  In this paper, we take early steps in (1) detecting EUV photons in ensembles of high z (3-3.5) LAE galaxies by building a large, exceptionally clean sample set from the HETDEX Data Release 1. Future papers will subselect from this data set and combine with archival photometry to explore correlations between directly and indirectly measurable properties (z, LyA EW, SiII absorption fetures, SED fit SFR, etc) and aggregate LyContinuum levels.}

{ \color{red} todo: define LyC as 880-910 like Steidel}

\section{Observations} \label{sec:obs}

TBD basics on observations (dates, sky regions, num fibers, etc)

\section{Data Reduction} \label{sec:reduction}

TBD basic data reduction ... reference instead a HETDEX paper?

{ \color{red} todo: Section or Subsection on key ELiXer features (PLAE/POII, catalog integration, aperture magnitudes) ?}

\subsection{Emission Line eXplorer} \label{sec:elixer}
{ \color{red} todo: maybe should be a subsection under Data Reduction? Hit the key points of ELiXer ... visual overview, imaging integration, PLAE/POII (cite Leung 2017) and source of the continuum estimates .... note the non-uniformity of the photometry and ELiXer's aperture magnitude }


\section{Data Selection} \label{sec:selection}

As it is absolutely critical that the data sample is free of mis-identifications and (especially insidious) faint continuum interlopers along the line of sight. Strict selection criteria were applied (described below in the order of application), and, because the more automated classification processes were not yet available, each potential detection was manually inspected before being included in the data set. As such, the selection criteria were set not only to isolate a clean sample, but also to keep the number of detections manageable for the bottleneck of manual examination. As the full HDR1 emission line catalog contains 690,868 detections (approximately 1/2 spurious and 1/4 not Ly$\alpha$) and the sample for this work was reduced to 271, certainly most valid LAEs were left "on the table" due to these highly limiting (but necessary) restrictions. 

The initial sample selection sought to eliminate obvious contaminants, spurious detections, and undesirable LAEs (specifically AGN with obvious continuum).

todo: go through each selection and provide pre/post selection counts and justification for the value used

The first step in reducing the potential LAE candidates from the set of all candidate detections in HDR1 was a simple threshold filter over several criteria. 

\subsection{Emission Wavelength}
Perhaps the easiest selection criteria to impose was the emission wavelength cut. Since we were interested in Lyman Continuum (880-910$\AA$) and the HETDEX observable wavelength range is 3470-5520 $\AA$, the minimum observed emission line wavelength was selected as 4860 $\AA$ (or z $\approx$ 3 for Ly$\alpha$) so that the blue end of our targeted Lyman Continuum range will be captured by the VIRUS spectrographs allowing for some small variation in each detector's wavelength range and avoid the edgemost pixels. This cut eliminated $\sim$ 75\% of all detections.

\subsection{Signal to Noise Ratio}
Although there is good recovery down to a SNR of approximately 5.2  { \color{red} todo: cite future? HETDEX paper of junk vs good recovery in Chenxu/Erin simulations}, in an effort to keep this preliminary manual examination sample manageable, a SNR minimum threshold of 6.5 was enforced. This reduced the sample size by another 61\%.

\subsection{$\chi^{2}$ Model Fit}
Closely related to the SNR cut, the $\chi^{2}$ limit of 2.0 was imposed to remove most detections of dubious quality, keeping only those detections with essentially single Gaussian line profiles within 50 $\AA$ of the line center.  { \color{red} todo: confirm the 50 AA ... might have been 40?} { \color{red} todo: also cite future? HETDEX paper of junk vs good recovery in Chenxu/Erin simulations}. This removed an additional 34\% of detections.

\subsection{Continuum Level}
As a likely indicator that the host is not an LAE (excepting quasars and other AGN), any detections with strong continuum observed in the HETDEX spectra (where "strong" is defined as outside of -1.0 to 2.0 $\times 10^{-17} erg\ s^{-1}\ cm^{-2}$ with the negative range included to handle poor fitting and over subtraction) were excluded. This also removed any bright continuum LAEs (again, some AGN), but for the purpose of exploring the more common population of galaxies, this was desirable and whittled the sample down by another 26\%. 

\subsection{Emission Line Width}
To remove charge traps, stuck pixels, hot columns or other non-astrophysical spikes in the data counts masquerading as signals, we imposed a minimum width ($\sigma gt 1.85$ ) to the Gaussian fit on the detection. Very few such false detections passed the initial HETDEX detection search, so this cut excluded less than 2\%. 

To this point, the total number of candidate detections has fallen to less than 5\% of the full set, but still numbers more than 30,000. While some measurement and reduction artifacts remain, most of these represent real astrophysical sources, though perhaps only 1/3 are Ly$\alpha$, so the next steps were focused on line classification. 

{ \color{red} todo: technically, an upper limit was also set, but it had zero impact ... and in the actual run, it was not included in the query}


\subsection{P(LAE)/P(OII)}

As described earlier in section \ref{sec:elixer} and in {\color{red} todo cite Andrew 2017} the principle automated classification mechanism employed by HDR1 was the Bayesian P(LAE)/P(OII) ratio. Since the continuum estimate was provided from multiple sources and each was sub-optimal and computed separately, inclusion in the final data set required that each available measure of P(LAE)/P(OII) agree. To be considered for this sample, all detections must have had valid P(LAE)/P(OII) calculations for the equivalent widths originating from the HETDEX spectra continuum estimate and the imaging fixed aperture continuum estimate. If either ratio was unavailable (for any reason), the detection was excluded. Not every detection had exactly one photometric catalog counterpart (those with ambiguous counterparts were removed in a later step), but where a photometric counterpart was identified, the P(LAE)/P(OII) ratio computed for its reported g-band or r-band magnitude was required to agree with the other two calculations. For the purpose of this work, ratios greater than 5 were considered probable LAE candidates (vs OII candidates).

After applying this criterion, 77\% of the remaining candidates were removed from the same for one or more of the P(LAE)/P(OII) ratios calculated below the threshold. This does not necessarily mean that a rejected candidate is not an LAE (as is discussed later in section \ref{sec:discussion}), only that the calculations failed to all agree that the candidate was likely an LAE.

\subsection{Catalog Closest Match}
Detections whose closest imaging catalog match (inside 3") was farther than 1" from the HETDEX weighted sky position was assumed to have possible issues with the astrometry. This is admittedly a weak assumption as the HDR1 astrometry was expected to be better than 1" (near 0.5") {\color{red} todo cite future HETDEX paper? ... after the number of IFUs reached 20 or so?} and the astrometry of the catalogs was not verified nor was the spatial extent of the catalog object or the HETDEX detection taken into consideration. Nevertheless, as the imaging catalogs played an important role in source identification, this potential discrepancy was mitigated by excluding such candidates from the final sample, removing 58\% of what remained.


\subsection{Catalog Neighborhood}  \label{sec:catalog_neighbors}
In an effort to eliminate source confusion (even if all nearby sources are likely LAEs (potentially interacting) and the inclusion of light from other (bright and/or spatially extended) sources, we next examined the (on-sky) projected neighborhood around each target and reject any detections that have one or more 'bad neighbors'. We first eliminated any (point source like) detection with a neighboring source brighter than 26th magnitude within 5", then reduce the magnitude and the separations, rejecting detections with neighbors fainter than 26 inside 4", fainter than 27 inside 3" and fainter than 28 inside 2". This removed 2\% of the remaining candidates.


\subsection{Nearby HETDEX Detections}
To reduce any line of sight biases (the over contribution of sources along lucky (or unlucky) lines of sight with abnormally low (or high) encounters with intervening neutral Hydrogen) no emission line detection within 5" of any other detection was allowed into the sample. The selection was effectively random as the second to nth subsequent detection was omitted based on the order in which they were imported (loosely in observation date order) and not on any characteristic (such as SNR or equivalent width). This condition reduced the sample by another 8\%.\\

The sample has been reduced to only 1828 candidate detections (some 0.26\% of the full HDR1 catalog) and the final sample selection was conducted via individual visual inspections of the corresponding ELiXer reports.

\subsection{Photometric Imaging}
At the time of this writing, approximately 1/3 of the HDR1 detections have not been matched with photometric imaging. Since continuum estimation from imaging is crucial in helping discriminate between Ly$\alpha$ and the primary contaminant ([OII]) as well as providing a manual check on source confusion from line of sight or proximate interlopers, all detections, regardless of their other attributes, without imaging were removed from consideration. This did not remove candidate detections that did not have a corresponding catalog detection (as the depth of the imaging was variable and the conditions for catalog inclusion were not uniform), but removed only those candidates without an overlapping image or where the image was corrupt, masked, incomplete, or otherwise poor. This eliminated 609 (33\%) of the remaining sample.

 \subsection{Missing Data}
 Any ELiXer reports that had missing data elements (for any reason ... missing reduction data, read error, communication or memory error during report generation, etc) was excluded. This accounted for the removal of another 101 detecions. It should be noted that there was an ingestion error (probably an I/O issue at the time HDR1 was assembled) with two observations that happened to be included in this reduced data set and all the detections from those two observations were the source of these removals. 
 
\subsection{Edge Fibers}
If too much of a detections light falls off the edge of an IFU, it becomes difficult or impossible to correctly model the astrometric solution and obtain a proper determination of the flux (accounting for lost light, etc). Of particular issue for this work is the loss of shorter-wavelength light due to DAR (which would fall off the edge of the detector). Without an accurate position, we cannot match the detection to the correct counterpart in a photometric catalog and without a correct spectra we cannot apply the various selection criteria or make a justifiably accurate classification. While it may be possible to correct these issues for specific observations depending on the orientation of the instrument and how much light and position information is lost, for the limited purpose and scope of this paper, any detection whose most central (highest weighted) lay on the edge of its IFU was removed from consideration. This condition eliminated 97 detections. 

\subsection{Non-Ly$\alpha$ Sources}
Any detections that were clearly not Ly$\alpha$ as generally evidenced by multiple emission lines that could not be consistent with a Ly$\alpha$ classification were removed. This comprised 77 detection that mostly consisted of resolved low-z galaxies where the HETDEX selected line was not [OII] (often an OIII or H$\beta$ line whose equivalent width was large enough to produce a large P(LAE)/P(OII) ratio when the line was tested as if it were [OII]).
{ \color{red} todo: show examples in appendix?}.  



\subsection{Bad Neighbors and Superimposed Spectra}
As with the step to remove detections with multiple nearby catalog neighbors in section \ref{sec:catalog_neighbors}, the associated imaging for each remaining candidate detection was examined for the presence of one or more nearby neighbors (inside 3") or for nearby large (extended) foreground objects that could be contributing flux (emission line or continuum) to the weighted 1D spectra. This examination removed an additional 64 candidates.
{ \color{red} todo: show examples in appendix?}


\subsection{Miscellaneous Rejections}
The final step in vetting the sample was a manual examination of the ELiXer detection report in which the 2D fiber cutouts, pixel flats, CCD region, 1D spectra fit, and full 1D spectra were reviewed for any disqualifying conditions (bad pixel flat, superimposed interference pattern, incompletely cleaned cosmic ray strikes, questionable line fit, etc). This last sweep removed 610 candidates, leaving a final data set of 270 "clean" detections { \color{red} todo: list here or in appendix? the detectID, name, RA, Dec}.  


\subsection{Sky Fibers Selection}

Lastly, in order to assert a positive detection above sky noise, it was necessary to identify and average over a large number of "blank" sky fibers. The initial sample of sky fibers was taken as the set of all fibers that shared a VIRUS Amplifier (for the same exposure) with one or more of the fibers included in a detection. That sample was then reduced to include only those fibers that did not exhibit obvious continuum or emission by removing from the set any fiber with one or more 2$\AA$ wide bins with flux exceeding $3.0\times 10^{-17} erg\ s^{-1}\ cm^{-2}$ or one or more 500$\AA$ wide bins with mean flux outside $\pm 1.0 \times 10^{-17} erg\ s^{-1}\ cm^{-2}$. These selection criteria yielded 93090 sky fibers.  

\section{Analysis and Results} \label{sec:analysis}

TBD description of analysis (shifting, stacking, etc) use of biweight vs mean vs median

{ \color{red} todo: progression of averaged spectra plots with increasing restriction on sample, show LyC fall off, etc .. over plot Steidel}


\begin{figure}[H]
\plotone{biweight_plot.png}
\caption{Biweight unbinned plot\label{fig:biweight_unbinned}}
\end{figure}

\begin{figure}[H]
\plotone{post_biweight_bin.png}
\caption{Biweight binned plot\label{fig:general}}
\end{figure}

\section{Discussion and Future Work} \label{sec:biweight_binned}
TBD brief discussion and future todos

{ \color{red} todo: Ly series and other lines appear to show a small offset to the blue, possibly indicative of LyA velocity offset, but would want much more data to identify and quantify}

{ \color{red} todo: discussion point, this is early data and classification is primitive ... HDR1 to inform and shape automated classification (machine learning, Bayesian ratio thresholds,etc) ... we do not know the probable fraction of contamination, etc; improvements to sky subtraction (can be wiping out true continuum), but is pretty good for local (to amp) sky}

{ \color{red} todo: discussion point, may have to manually inspect each detection ... mechanical turk?}

{ \color{red} todo: discussion point, LyA strength (EW) alone is not a good predictor of LyC as LyA is a complex function of physics and geometry there may even be an anti-correlation of LyC with LyA SN? If LyA strong could mean lots of neutral H and self shielding so little LyC escape ... but low sn, little hydrogen so little LyA produced but less shielding and relatively more LyC escapes?  So, for LyC detection might want lots of low sn and low EW LyA
Right now that's hard since need to manually examine all }

Potential contamination excluded (misidentification of emission line and faint continuum LOS interlopers), noise is the main limiting factor in our Lyman Continuum measurement. Future enhancements to the HETDEX pipeline should help address this issue by providing more readied access to larger LAE samples with both improved SNR and a lower viable SNR limit.

Todo: sky subtraction, machine learning, Bayesian ratio thresholds, additional imagaing catalogs with uniform magnitude corrections, 

%% If you wish to include an acknowledgments section in your paper,
%% separate it off from the body of the text using the \acknowledgments
%% command.
\acknowledgments

TBD acknowledgements

\pagebreak
\appendix


\section{Appendix - HDR1 Accepted Detections}

\begin{center}
\begin{longtable}{ |c|c|c|c| } 
 \hline
\textbf{DetectID} & \textbf{Detect Name} & \textbf{RA (degrees)} & \textbf{Dec(degrees)} \\
 \hline

1000004756 & HETDEX J012407.78-000232.72w+5055.23 & 21.032412 & -0.042422 \\
1000005057 & HETDEX J012405.50-000423.07w+4869.04 & 21.022919 & -0.073074 \\
1000017083 & HETDEX J100021.73+022903.33w+5210.60 & 150.090530 & 2.484257 \\
1000017312 & HETDEX J100015.34+022809.92w+5425.37 & 150.063900 & 2.469423 \\
1000017927 & HETDEX J100031.34+021652.35w+4985.25 & 150.130570 & 2.281208 \\
1000017993 & HETDEX J100020.43+021735.02w+5046.14 & 150.085110 & 2.293061 \\
1000021258 & HETDEX J123555.00+621521.79w+5246.34 & 188.979160 & 62.256054 \\
1000038877 & HETDEX J141716.71+522310.47w+4925.91 & 214.319630 & 52.386242 \\
1000041942 & HETDEX J141824.62+524421.25w+5139.63 & 214.602580 & 52.739235 \\
1000045021 & HETDEX J142051.34+523424.83w+5277.47 & 215.213930 & 52.573563 \\
1000045941 & HETDEX J011429.46-000127.82w+5012.85 & 18.622740 & -0.024394 \\
1000048312 & HETDEX J015315.45-000037.42w+4995.69 & 28.314386 & -0.010395 \\
1000048558 & HETDEX J003904.41+000149.49w+4949.58 & 9.768383 & 0.030414 \\
1000048736 & HETDEX J010756.47-000005.47w+5198.91 & 16.985281 & -0.001519 \\
1000049437 & HETDEX J013653.73+000423.73w+5190.16 & 24.223871 & 0.073257 \\
1000051275 & HETDEX J014125.03-000424.03w+5086.95 & 25.354277 & -0.073342 \\
1000051278 & HETDEX J014150.40-000115.51w+4999.80 & 25.459990 & -0.020975 \\
1000051661 & HETDEX J003836.21-000543.04w+5022.44 & 9.650857 & -0.095288 \\
1000052545 & HETDEX J003828.66-000254.29w+5426.45 & 9.619396 & -0.048414 \\
1000053595 & HETDEX J005448.61-000016.54w+5088.66 & 13.702525 & -0.004594 \\
1000060448 & HETDEX J015024.77+000000.48w+5331.71 & 27.603226 & 0.000134 \\
1000061235 & HETDEX J011130.44+000154.87w+4887.29 & 17.876816 & 0.031909 \\
1000064188 & HETDEX J012616.20+000309.32w+4978.58 & 21.567503 & 0.052590 \\
1000064310 & HETDEX J015418.88+000244.11w+4910.69 & 28.578669 & 0.045586 \\
1000064529 & HETDEX J020553.47+000122.59w+5380.98 & 31.472792 & 0.022941 \\
1000065527 & HETDEX J012421.47+000431.29w+5293.70 & 21.089462 & 0.075359 \\
1000066121 & HETDEX J003628.88+000330.61w+4958.41 & 9.120334 & 0.058502 \\
1000066284 & HETDEX J012542.16+000225.40w+4918.00 & 21.425665 & 0.040388 \\
1000066583 & HETDEX J015132.28-000305.16w+5264.99 & 27.884502 & -0.051432 \\
1000068539 & HETDEX J020432.43-000247.95w+4914.62 & 31.135130 & -0.046652 \\
1000069676 & HETDEX J005931.21-000445.14w+4923.98 & 14.880028 & -0.079205 \\
1000070131 & HETDEX J012735.71-000444.25w+4924.13 & 21.898775 & -0.078957 \\
1000070849 & HETDEX J014526.58+000253.89w+4951.06 & 26.360770 & 0.048302 \\
1000072012 & HETDEX J004522.98+000312.32w+4914.48 & 11.345748 & 0.053423 \\
1000072548 & HETDEX J005249.34+000400.47w+5016.61 & 13.205600 & 0.066796 \\
1000072554 & HETDEX J005243.68+000529.88w+4941.89 & 13.181988 & 0.091634 \\
1000072610 & HETDEX J005316.51+000216.18w+5288.39 & 13.318801 & 0.037829 \\
1000073318 & HETDEX J014842.30-000352.37w+5156.36 & 27.176245 & -0.064548 \\
1000073562 & HETDEX J004619.61+000025.82w+5268.63 & 11.581712 & 0.007173 \\
1000076526 & HETDEX J011822.21-000057.39w+5309.47 & 19.592545 & -0.015942 \\
1000081342 & HETDEX J005809.75+000049.14w+5377.64 & 14.540643 & 0.013651 \\
1000086864 & HETDEX J014000.70+000141.84w+5010.19 & 25.002907 & 0.028288 \\
1000087062 & HETDEX J014009.53+000538.33w+5319.15 & 25.039707 & 0.093980 \\
1000087799 & HETDEX J014005.40+000112.04w+4953.98 & 25.022480 & 0.020012 \\
1000088685 & HETDEX J012109.71+000256.94w+5308.30 & 20.290442 & 0.049149 \\
1000088710 & HETDEX J012050.08-000229.78w+5023.37 & 20.208668 & -0.041605 \\
1000089312 & HETDEX J005036.14+000253.26w+4921.98 & 12.650578 & 0.048129 \\
1000094147 & HETDEX J013050.62-000308.46w+5364.16 & 22.710897 & -0.052351 \\
1000094170 & HETDEX J013042.73+000329.65w+4978.43 & 22.678024 & 0.058237 \\
1000097747 & HETDEX J004130.05+000028.53w+5148.61 & 10.375208 & 0.007924 \\
1000104984 & HETDEX J100001.97+020953.82w+5134.00 & 150.008210 & 2.164951 \\
1000116655 & HETDEX J120726.66+511230.02w+5019.59 & 181.861070 & 51.208340 \\
1000134071 & HETDEX J134738.03+510955.27w+4879.27 & 206.908460 & 51.165350 \\
1000167951 & HETDEX J131218.95+511201.98w+5133.48 & 198.078960 & 51.200550 \\
1000168585 & HETDEX J133828.28+510220.02w+5066.64 & 204.617830 & 51.038895 \\
1000184724 & HETDEX J142930.20+511208.26w+4917.00 & 217.375840 & 51.202293 \\
1000192464 & HETDEX J120833.67+510508.56w+4905.41 & 182.140300 & 51.085712 \\
1000229997 & HETDEX J134934.44+511259.86w+4916.06 & 207.393500 & 51.216630 \\
1000230411 & HETDEX J120857.90+510047.52w+4894.00 & 182.241240 & 51.013200 \\
1000231088 & HETDEX J123222.38+511202.53w+5055.15 & 188.093230 & 51.200703 \\
1000243319 & HETDEX J132404.49+511242.97w+4878.59 & 201.018720 & 51.211937 \\
1000247999 & HETDEX J125519.85+510935.22w+5032.79 & 193.832720 & 51.159782 \\
1000250027 & HETDEX J141659.26+511037.73w+4861.60 & 214.246930 & 51.177147 \\
1000250376 & HETDEX J121949.37+505938.30w+5355.50 & 184.955690 & 50.993973 \\
1000250481 & HETDEX J133530.94+510104.17w+4963.98 & 203.878900 & 51.017826 \\
1000250529 & HETDEX J140339.61+510921.47w+4885.37 & 210.915020 & 51.155964 \\
1000251464 & HETDEX J122618.04+511135.53w+5283.67 & 186.575160 & 51.193203 \\
1000253348 & HETDEX J141444.85+510734.16w+5442.80 & 213.686870 & 51.126156 \\
1000253409 & HETDEX J144035.49+511414.21w+5334.67 & 220.147870 & 51.237280 \\
1000253624 & HETDEX J130633.62+511214.45w+5006.26 & 196.640080 & 51.204014 \\
1000253802 & HETDEX J142450.36+510311.27w+4896.06 & 216.209820 & 51.053130 \\
1000254149 & HETDEX J132652.69+510330.57w+5322.61 & 201.719530 & 51.058490 \\
1000254189 & HETDEX J135541.82+510356.38w+5073.69 & 208.924260 & 51.065662 \\
1000254195 & HETDEX J135610.39+511449.00w+5138.38 & 209.043270 & 51.246944 \\
1000254461 & HETDEX J132527.67+511210.65w+4902.17 & 201.365310 & 51.202957 \\
1000258853 & HETDEX J144407.66+510031.89w+5331.33 & 221.031940 & 51.008858 \\
1000276882 & HETDEX J121108.85+511909.88w+5317.97 & 182.786880 & 51.319412 \\
1000278050 & HETDEX J141209.80+511101.43w+5358.04 & 213.040850 & 51.183730 \\
1000278471 & HETDEX J143744.52+511243.06w+5286.15 & 219.435520 & 51.211960 \\
1000304488 & HETDEX J140352.59+511544.47w+4899.32 & 210.969120 & 51.262352 \\
1000322135 & HETDEX J140452.62+511337.04w+4880.39 & 211.219270 & 51.226955 \\
1000322268 & HETDEX J140457.65+512345.34w+5002.62 & 211.240190 & 51.395927 \\
1000328979 & HETDEX J131050.91+512304.66w+5158.66 & 197.712140 & 51.384630 \\
1000329549 & HETDEX J135847.21+510908.14w+4936.55 & 209.696720 & 51.152260 \\
1000330945 & HETDEX J121812.64+512225.21w+4962.94 & 184.552670 & 51.373670 \\
1000331257 & HETDEX J124632.30+512406.83w+5012.86 & 191.634600 & 51.401897 \\
1000331726 & HETDEX J133450.87+511734.12w+4985.22 & 203.711940 & 51.292812 \\
1000332153 & HETDEX J135852.81+512256.79w+5050.64 & 209.720050 & 51.382442 \\
1000332223 & HETDEX J135826.31+512226.31w+5143.20 & 209.609620 & 51.373974 \\
1000332426 & HETDEX J142303.72+510249.45w+4966.88 & 215.765520 & 51.047070 \\
1000336294 & HETDEX J122207.29+511852.14w+5024.76 & 185.530380 & 51.314484 \\
1000336320 & HETDEX J124344.15+512809.99w+4894.80 & 190.933940 & 51.469440 \\
1000336520 & HETDEX J124323.16+511652.02w+5223.99 & 190.846480 & 51.281116 \\
1000336559 & HETDEX J134010.31+511032.14w+4900.86 & 205.042970 & 51.175594 \\
1000336574 & HETDEX J133959.24+510032.26w+5207.83 & 204.996830 & 51.008960 \\
1000342145 & HETDEX J122906.61+512732.48w+4949.33 & 187.277560 & 51.459023 \\
1000342176 & HETDEX J122845.88+511953.29w+4943.05 & 187.191160 & 51.331470 \\
1000342212 & HETDEX J122911.67+511742.20w+4894.14 & 187.298650 & 51.295055 \\
1000344112 & HETDEX J123735.49+510928.75w+5210.94 & 189.397860 & 51.157986 \\
1000345118 & HETDEX J134858.49+511703.28w+5334.96 & 207.243710 & 51.284245 \\
1000347199 & HETDEX J135415.21+512236.32w+5413.56 & 208.563370 & 51.376755 \\
1000347229 & HETDEX J135356.04+511446.17w+5165.61 & 208.483500 & 51.246160 \\
1000347495 & HETDEX J135328.56+511802.08w+4914.30 & 208.368990 & 51.300580 \\
1000347507 & HETDEX J135335.80+511724.48w+5070.18 & 208.399150 & 51.290134 \\
1000348150 & HETDEX J141741.23+512242.72w+5448.49 & 214.421800 & 51.378532 \\
1000352600 & HETDEX J131820.46+511546.39w+4937.21 & 199.585250 & 51.262886 \\
1000352896 & HETDEX J133901.02+512215.42w+5114.50 & 204.754240 & 51.370950 \\
1000352922 & HETDEX J133928.56+511503.42w+5171.95 & 204.869000 & 51.250950 \\
1000353117 & HETDEX J133941.73+512311.17w+5035.20 & 204.923890 & 51.386436 \\
1000353638 & HETDEX J143111.57+511006.34w+5413.01 & 217.798220 & 51.168427 \\
1000356353 & HETDEX J135527.65+512728.13w+5044.19 & 208.865190 & 51.457813 \\
1000356387 & HETDEX J135532.34+511634.63w+5094.81 & 208.884770 & 51.276287 \\
1000356962 & HETDEX J141953.28+511831.23w+4880.74 & 214.972000 & 51.308674 \\
1000359114 & HETDEX J130402.77+511726.74w+4877.13 & 196.011550 & 51.290760 \\
1000367720 & HETDEX J143655.46+513109.36w+4888.59 & 219.231100 & 51.519268 \\
1000368855 & HETDEX J133600.88+512446.86w+5042.74 & 204.003680 & 51.413017 \\
1000369736 & HETDEX J142613.88+512240.99w+5249.95 & 216.557850 & 51.378050 \\
1000372660 & HETDEX J142550.38+512538.94w+4910.09 & 216.459920 & 51.427483 \\
1000373711 & HETDEX J142533.48+511542.21w+4984.27 & 216.389500 & 51.261726 \\
1000373824 & HETDEX J142615.34+512801.57w+4912.53 & 216.563920 & 51.467102 \\
1000391803 & HETDEX J005141.48-001249.95w+5046.90 & 12.922834 & -0.213876 \\
1000393301 & HETDEX J005156.46-000911.94w+5098.39 & 12.985264 & -0.153318 \\
1000394034 & HETDEX J021300.05-001108.47w+5021.31 & 33.250190 & -0.185687 \\
1000394048 & HETDEX J021241.38-000903.91w+5307.30 & 33.172410 & -0.151087 \\
1000395556 & HETDEX J013837.98-001322.15w+5136.47 & 24.658262 & -0.222819 \\
1000395689 & HETDEX J013819.44-000818.76w+4863.62 & 24.581017 & -0.138545 \\
1000395988 & HETDEX J020205.62-001122.07w+4876.34 & 30.523432 & -0.189464 \\
1000396060 & HETDEX J020242.44-000653.69w+5087.31 & 30.676838 & -0.114915 \\
1000396061 & HETDEX J020223.06-001305.70w+5229.66 & 30.596088 & -0.218249 \\
1000396577 & HETDEX J004051.74+000934.81w+5301.48 & 10.215600 & 0.159669 \\
1000396879 & HETDEX J013929.58-001420.80w+5216.67 & 24.873245 & -0.239111 \\
1000397651 & HETDEX J005331.96+000531.49w+5369.88 & 13.383183 & 0.092081 \\
1000401765 & HETDEX J003805.44+000437.32w+5428.15 & 9.522676 & 0.077033 \\
1000402432 & HETDEX J003820.03-000119.99w+4974.68 & 9.583473 & -0.022219 \\
1000403125 & HETDEX J003809.35+000317.23w+5336.81 & 9.538977 & 0.054787 \\
1000403141 & HETDEX J003720.86-000218.00w+5148.53 & 9.336936 & -0.038334 \\
1000403176 & HETDEX J003806.43+000451.68w+5277.05 & 9.526811 & 0.081021 \\
1000404563 & HETDEX J003451.52+000605.50w+4929.72 & 8.714687 & 0.101529 \\
1000404745 & HETDEX J010253.17+000400.84w+4889.98 & 15.721552 & 0.066900 \\
1000404829 & HETDEX J010235.01-000150.68w+5154.65 & 15.645863 & -0.030745 \\
1000404932 & HETDEX J012830.54-000338.60w+5212.53 & 22.127243 & -0.060723 \\
1000406024 & HETDEX J004434.08-000021.82w+5053.89 & 11.141996 & -0.006061 \\
1000408159 & HETDEX J004500.22-001016.77w+4891.35 & 11.250896 & -0.171324 \\
1000408272 & HETDEX J004419.45-001501.99w+5205.71 & 11.081056 & -0.250553 \\
1000408598 & HETDEX J010931.79-000033.66w+5277.21 & 17.382475 & -0.009351 \\
1000409313 & HETDEX J020116.85+000119.53w+4981.65 & 30.320198 & 0.022093 \\
1000413478 & HETDEX J003652.86+000525.93w+5162.24 & 9.220230 & 0.090537 \\
1000414346 & HETDEX J015400.21-000547.38w+5260.42 & 28.500889 & -0.096494 \\
1000414827 & HETDEX J021738.97-000255.69w+4927.05 & 34.412376 & -0.048804 \\
1000415723 & HETDEX J015205.68+001849.09w+4947.95 & 28.023676 & 0.313637 \\
1000415813 & HETDEX J015123.46+001643.62w+5260.59 & 27.847757 & 0.278784 \\
1000416005 & HETDEX J021511.80+002206.73w+4868.34 & 33.799152 & 0.368537 \\
1000418041 & HETDEX J021353.61-000235.26w+4916.07 & 33.473360 & -0.043127 \\
1000418069 & HETDEX J021415.76-000512.09w+4966.06 & 33.565662 & -0.086692 \\
1000418880 & HETDEX J020847.93+001940.20w+5110.08 & 32.199715 & 0.327834 \\
1000418999 & HETDEX J020845.70+001954.55w+5437.47 & 32.190437 & 0.331820 \\
1000420311 & HETDEX J013015.03-000529.26w+5190.29 & 22.562634 & -0.091460 \\
1000420368 & HETDEX J012942.79-000733.90w+4861.71 & 22.428300 & -0.126084 \\
1000420501 & HETDEX J015743.05-000604.22w+5120.03 & 29.429358 & -0.101173 \\
1000420887 & HETDEX J015747.97-000759.60w+5133.61 & 29.449875 & -0.133223 \\
1000421572 & HETDEX J003439.43+000533.94w+4903.95 & 8.664275 & 0.092762 \\
1000421651 & HETDEX J010025.47+001243.15w+4871.89 & 15.106141 & 0.211986 \\
1000421717 & HETDEX J010037.09+001327.01w+5107.81 & 15.154559 & 0.224169 \\
1000422037 & HETDEX J012136.80+001840.90w+5189.33 & 20.403350 & 0.311362 \\
1000425703 & HETDEX J004620.31-000547.95w+5053.91 & 11.584635 & -0.096654 \\
1000428533 & HETDEX J011021.21+000653.18w+5220.14 & 17.588362 & 0.114773 \\
1000431306 & HETDEX J013259.38-001258.65w+5001.42 & 23.247402 & -0.216292 \\
1000431462 & HETDEX J015833.68-000947.42w+5217.66 & 29.640352 & -0.163172 \\
1000431722 & HETDEX J015912.84-000304.09w+5193.20 & 29.803516 & -0.051136 \\
1000433446 & HETDEX J003943.10-000103.05w+5099.13 & 9.929599 & -0.017515 \\
1000433855 & HETDEX J003949.18-000012.47w+5308.45 & 9.954912 & -0.003463 \\
1000434598 & HETDEX J014450.39-001139.29w+5167.16 & 26.209963 & -0.194248 \\
1000434937 & HETDEX J020753.49-000839.11w+5390.21 & 31.972887 & -0.144197 \\
1000435002 & HETDEX J020713.17-001133.49w+4942.02 & 31.804869 & -0.192635 \\
1000435391 & HETDEX J003413.48+001516.27w+4993.07 & 8.556173 & 0.254520 \\
1000435393 & HETDEX J003400.66+001717.44w+5177.47 & 8.502763 & 0.288178 \\
1000436185 & HETDEX J012057.74+000844.09w+5151.15 & 20.240572 & 0.145580 \\
1000436284 & HETDEX J012137.82+000715.52w+5289.01 & 20.407597 & 0.120979 \\
1000436389 & HETDEX J014602.13+000521.41w+4917.63 & 26.508883 & 0.089280 \\
1000436582 & HETDEX J014621.61+001005.87w+5270.13 & 26.590025 & 0.168297 \\
1000437999 & HETDEX J013215.69-000506.76w+4956.21 & 23.065360 & -0.085212 \\
1000438347 & HETDEX J021529.72-000611.34w+5096.21 & 33.873820 & -0.103150 \\
1000439235 & HETDEX J014717.79+002216.22w+5070.64 & 26.824130 & 0.371171 \\
1000439341 & HETDEX J021807.20+000733.54w+4865.56 & 34.529980 & 0.125983 \\
1000440999 & HETDEX J005221.79-000120.14w+5256.46 & 13.090799 & -0.022261 \\
1000443961 & HETDEX J020125.29-000517.87w+4974.08 & 30.355390 & -0.088297 \\
1000444524 & HETDEX J005117.80+000757.00w+5211.69 & 12.824154 & 0.132501 \\
1000444728 & HETDEX J011725.62+002144.73w+5018.69 & 19.356731 & 0.362426 \\
1000449454 & HETDEX J015235.00+002320.14w+5109.91 & 28.145832 & 0.388929 \\
1000450858 & HETDEX J003448.27-001342.23w+4981.26 & 8.701112 & -0.228398 \\
1000454589 & HETDEX J003553.99+002557.73w+4986.42 & 8.974961 & 0.432702 \\
1000454945 & HETDEX J003617.51+001215.25w+5036.69 & 9.072959 & 0.204236 \\
1000456342 & HETDEX J004647.47+000530.99w+5399.69 & 11.697803 & 0.091942 \\
1000457878 & HETDEX J010920.29+002227.77w+4891.11 & 17.334530 & 0.374381 \\
1000458598 & HETDEX J013324.20+001558.69w+5304.22 & 23.350828 & 0.266304 \\
1000458967 & HETDEX J013326.57+001659.39w+4973.48 & 23.360706 & 0.283164 \\
1000459535 & HETDEX J015722.07+002106.48w+5162.08 & 29.341978 & 0.351801 \\
1000461425 & HETDEX J003811.27+000716.22w+5013.79 & 9.546971 & 0.121173 \\
1000461625 & HETDEX J003746.65+000619.57w+4934.71 & 9.444364 & 0.105435 \\
1000461761 & HETDEX J010242.14+000812.37w+5463.64 & 15.675583 & 0.136770 \\
1000462125 & HETDEX J012626.60+000832.04w+5406.98 & 21.610838 & 0.142232 \\
1000462841 & HETDEX J021406.76+001235.25w+5066.83 & 33.528160 & 0.209791 \\
1000464270 & HETDEX J014006.25+000947.02w+4905.06 & 25.026041 & 0.163062 \\
1000465593 & HETDEX J003718.71+001117.68w+5224.93 & 9.327952 & 0.188244 \\
1000465677 & HETDEX J003718.60+001112.58w+5223.19 & 9.327481 & 0.186829 \\
1000466402 & HETDEX J012724.15+001440.17w+5074.95 & 21.850622 & 0.244491 \\
1000466962 & HETDEX J015034.33+002411.12w+5011.43 & 27.643023 & 0.403090 \\
1000467248 & HETDEX J021414.08+001727.17w+5366.10 & 33.558685 & 0.290880 \\
1000467628 & HETDEX J021426.89+001335.23w+5268.19 & 33.612038 & 0.226454 \\
1000474499 & HETDEX J015223.78-001316.77w+5291.15 & 28.099085 & -0.221326 \\
1000475107 & HETDEX J003849.08-000830.99w+5236.13 & 9.704511 & -0.141942 \\
1000481242 & HETDEX J011523.09-001144.91w+5135.23 & 18.846207 & -0.195808 \\
1000483085 & HETDEX J004020.28+000721.54w+4871.40 & 10.084510 & 0.122650 \\
1000483090 & HETDEX J004011.21+001231.96w+5148.50 & 10.046718 & 0.208878 \\
1000483092 & HETDEX J004030.42+001147.84w+5328.11 & 10.126736 & 0.196621 \\
1000483126 & HETDEX J003947.83+000927.08w+4985.77 & 9.949280 & 0.157521 \\
1000483132 & HETDEX J003959.66+001351.84w+5233.00 & 9.998597 & 0.231066 \\
1000483138 & HETDEX J004024.03+000048.96w+5188.23 & 10.100104 & 0.013600 \\
1000485056 & HETDEX J021535.10+001843.74w+5409.33 & 33.896248 & 0.312149 \\
1000492686 & HETDEX J012936.22+001640.62w+4877.64 & 22.400936 & 0.277950 \\
1000494084 & HETDEX J003742.65-001658.84w+5228.80 & 9.427719 & -0.283011 \\
1000494435 & HETDEX J010013.97-000857.10w+5064.93 & 15.058214 & -0.149195 \\
1000496487 & HETDEX J021154.97-001507.37w+5057.77 & 32.979040 & -0.252048 \\
1000497658 & HETDEX J012916.86+001926.06w+4950.73 & 22.320267 & 0.323906 \\
1000499635 & HETDEX J021000.22+001449.93w+4904.76 & 32.500910 & 0.247202 \\
1000502769 & HETDEX J022249.86+001528.61w+5301.44 & 35.707733 & 0.257947 \\
1000513960 & HETDEX J004017.09-002302.44w+5137.35 & 10.071214 & -0.384011 \\
1000514278 & HETDEX J004015.70-001548.78w+4889.54 & 10.065428 & -0.263549 \\
1000515236 & HETDEX J013226.14+000700.91w+4962.98 & 23.108898 & 0.116920 \\
1000515404 & HETDEX J013143.68+000856.37w+5087.61 & 22.931993 & 0.148992 \\
1000515483 & HETDEX J013658.85+000958.52w+4936.64 & 24.245188 & 0.166256 \\
1000515498 & HETDEX J013647.11+000741.66w+5101.98 & 24.196306 & 0.128238 \\
1000523413 & HETDEX J100109.21+021627.65w+5170.92 & 150.288380 & 2.274346 \\
1000525523 & HETDEX J123827.65+505322.58w+4961.91 & 189.615220 & 50.889606 \\
1000525935 & HETDEX J125938.06+514402.46w+4866.56 & 194.908580 & 51.734016 \\
1000526166 & HETDEX J130028.03+514632.70w+4929.92 & 195.116780 & 51.775750 \\
1000528344 & HETDEX J015805.79+001452.05w+5000.66 & 29.524130 & 0.247793 \\
1000529268 & HETDEX J015838.59+001714.30w+5060.80 & 29.660786 & 0.287306 \\
1000530196 & HETDEX J021755.78-000152.57w+5014.26 & 34.482426 & -0.031269 \\
1000530206 & HETDEX J021824.13-000356.71w+5185.78 & 34.600536 & -0.065753 \\
1000531007 & HETDEX J021503.21+000318.15w+5304.60 & 33.763367 & 0.055043 \\
1000531108 & HETDEX J021436.84-000553.91w+5208.60 & 33.653510 & -0.098308 \\
1000531723 & HETDEX J003416.64+003051.92w+4971.54 & 8.569321 & 0.514422 \\
1000537398 & HETDEX J004009.37-001413.30w+5347.21 & 10.039058 & -0.237027 \\
1000538551 & HETDEX J003953.19-000817.96w+5013.26 & 9.971633 & -0.138321 \\
1000541093 & HETDEX J012951.93+003421.51w+4945.55 & 22.466364 & 0.572642 \\
1000547217 & HETDEX J015538.29-001323.85w+5424.57 & 28.909529 & -0.223291 \\
1000547692 & HETDEX J021651.52+000326.49w+4988.21 & 34.214650 & 0.057359 \\
1000547741 & HETDEX J021633.09+000544.68w+5013.60 & 34.137870 & 0.095745 \\
1000555783 & HETDEX J131402.15+512220.81w+4932.98 & 198.508960 & 51.372448 \\
1000560685 & HETDEX J004341.62+002800.36w+5182.35 & 10.923418 & 0.466766 \\
1000560946 & HETDEX J004249.12+003121.86w+5090.32 & 10.704664 & 0.522739 \\
1000562823 & HETDEX J014225.10+002648.63w+5163.42 & 25.604576 & 0.446842 \\
1000562908 & HETDEX J014206.27+003158.29w+4991.08 & 25.526125 & 0.532859 \\
1000574506 & HETDEX J134457.84+510159.09w+5174.33 & 206.241000 & 51.033080 \\
1000574583 & HETDEX J134428.21+505521.73w+5043.40 & 206.117550 & 50.922703 \\
1000575074 & HETDEX J140822.19+512359.83w+5046.37 & 212.092470 & 51.399952 \\
1000575319 & HETDEX J140802.68+513139.19w+5447.97 & 212.011150 & 51.527554 \\
1000596304 & HETDEX J020723.37+000817.53w+5080.84 & 31.847391 & 0.138204 \\
1000597832 & HETDEX J100103.84+021541.77w+5358.81 & 150.266000 & 2.261604 \\
1000603505 & HETDEX J124605.26+510720.37w+5242.44 & 191.521910 & 51.122326 \\
1000604716 & HETDEX J133344.32+511010.29w+4989.58 & 203.434680 & 51.169525 \\
1000625350 & HETDEX J122422.66+514117.31w+4878.54 & 186.094420 & 51.688140 \\
1000638057 & HETDEX J124733.96+511753.42w+5012.50 & 191.891480 & 51.298172 \\
1000638755 & HETDEX J131009.55+514105.30w+5052.94 & 197.539780 & 51.684807 \\
1000639246 & HETDEX J133527.90+511820.87w+5001.55 & 203.866240 & 51.305798 \\
1000646421 & HETDEX J134550.26+515037.23w+4924.12 & 206.459430 & 51.843674 \\
1000651097 & HETDEX J132015.73+505929.68w+5237.80 & 200.065550 & 50.991577 \\
1000676413 & HETDEX J123843.33+505105.50w+5116.22 & 189.680560 & 50.851530 \\
1000677166 & HETDEX J131815.08+513333.13w+4931.92 & 199.562820 & 51.559204 \\
 \hline
\end{longtable}
\end{center}

\newpage
\section{Appendix - ELiXer Examples}

\begin{figure}[H]
\plotone{1000167702.pdf}
\caption{ELiXer Report (NGC 4707). Example \label{fig:elixer_ngc_4707}}
\end{figure}

\begin{figure}[H]
\plotone{1000167702_wide.pdf}
\caption{ELiXer Report (NGC 4707). Zoomed out view clearly shows the spatially extended galaxy. Note that there are no accepted detections within 11 arcmin of NGC 4707.\label{fig:elixer_ngc_4707_wide}}
\end{figure}


\begin{figure}[H]
\plotone{1000395765.pdf}
\caption{ELiXer Report Example of Non-LAE. The emission line of the detection is OIII (5007 $\AA$) but passed the P(LAE)/P(OII) test for the spectrum estimated continuum and the aperture estimated continuum. You can see the aperture is in the halo of the galaxy. \label{fig:elixer_not_lae}}
\end{figure}

\begin{figure}[H]
\plotone{1000395765_wide.pdf}
\caption{ELiXer Report Example of Non-LAE. Zoomed out view clearly shows the spatially extended galaxy that is reported (center) outside of 3" but that would otherwise have failed the P(LAE)/P(OII) test.\label{fig:elixer_not_lae_wide}}
\end{figure}

\begin{thebibliography}{}
\bibitem[Bromm and Yoshida (2011)]{Bromm}
Bromm, V. and Yoshida, N. "The First Galaxies", arXiv:1102.4638v1 (2011)

\bibitem[Chisholm et al (2018)]{Chisholm}
Chisholm J., et al "Accurately predicting the escape fraction of ionizing photons using rest-frame ultraviolet absorption lines", arXiv:1803.03655v2 (2018)

\bibitem[Duncan and Conselice (2015)]{Duncan}
Duncan, K. and Conselice, C. "Powering reionization: assessing the galaxy ionizing photon budget at z $<$ 10", arXiv:1505.01846v1 (2015)

\bibitem[Gazagnes et al (2018)]{Gazagnes}
Gazagnes, S. et al "Neutral gas properties of Lyman continuum emitting galaxies: Column densities and covering fractions from UV absorption lines", arXiv:1802.06378v2 (2018)

\bibitem[Livermore and Finkelstein (2018)]{Livermore}
Livermore, R.C. and Finkelstein, S.L. "Directly Observing the Galaxies Likely Responsible for Reionization" arXiv:1604.06799v2 (2018)

\bibitem[Stark (2016)]{Stark}
Stark, D., "Galaxies in the First Billion Years After the Big Bang", Annu. Rev. Astron. Astrophys. 2016.54:761-803

\bibitem[Tsang (2017)]{Tsang}
Tsang, B. "Supernovae Made Way for Cosmic Reionization", Astrobites May 29, 2017

\bibitem[Yue et al (2018)]{Yue}
Yue, B. et al "On the Faint-End of the Galaxy Luminosity Function in the Epoch of Reionization: Updated Constraints from the HST Frontier Fields", arXiv:1711.05130v4 (2018)

\bibitem[Zackrisson et al (2017)]{Zackrisson}
Zackrisson, E., et al "The Spectral Evolution of the First Galaxies. III. Simulated James Webb Space Telescope Spectra of Reionization-Epoch Galaxies with Lyman Continuum Leakage", arXiv:1608.08217v2 (2017)

\end{thebibliography}



%% This command is needed to show the entire author+affilation list when
%% the collaboration and author truncation commands are used.  It has to
%% go at the end of the manuscript.
%\allauthors

%% Include this line if you are using the \added, \replaced, \deleted
%% commands to see a summary list of all changes at the end of the article.
%\listofchanges

\end{document}




% End of file `sample62.tex'.
